\documentclass[12pt]{article}

\usepackage{fouriernc}
\usepackage[T1]{fontenc}
\usepackage{amsmath}
\usepackage{amssymb}
\usepackage[utf8]{inputenc}
\usepackage[english]{babel}
\usepackage{multicol}
\usepackage[margin=0.5in]{geometry}

\setlength{\parindent}{0em}
\setlength{\parskip}{1em}

\newcommand{\curly}[1]{\left\{ #1 \right\}}
\newcommand{\round}[1]{\left( #1 \right)}
\newcommand{\hard}[1]{\left[ #1 \right]}
\newcommand{\C}[2]{
    \begin{pmatrix}
        #1 \\ #2
    \end{pmatrix}
}
\newcommand{\iif}{\Longleftrightarrow}

\title{Book of Proof Notes}
\author{Tim Harding}

\begin{document}
\begin{multicols*}{2}

\section*{Sets}

\begin{align*}
    A \subseteq B \quad \rightarrow \quad \forall a \in A, a \in B
\end{align*}
\begin{align*}
    A \times B = \curly{(a,b) : (a \in A) \wedge (b \in B)}
\end{align*}
\begin{align*}
    |A \times B| = |A| \times |B|
\end{align*}
\begin{align*}
    \mathcal{P}(A) = \curly{X : X \subseteq A}
\end{align*}
\begin{align*}
    |A| = n \quad \rightarrow \quad |\mathcal{P}(A)| = 2^n
\end{align*}
\begin{align*}
    A \cup B &= \curly{x : (x \in A) \vee (x \in B)} \\
    A \cap B &= \curly{x : (x \in A) \wedge (x \in B)} \\
    A - B &= \curly{x : (x \in A) \wedge (x \notin B)}
\end{align*}
\begin{align*}
    \bar{A} = U - A
\end{align*}
\begin{align*}
    \bigcup_{i = 1}^n A_i &= A_1 \cup A_2 \cup \cdots \cup A_n \\
    \bigcap_{i = 1}^n A_i &= A_1 \cap A_2 \cap \cdots \cap A_n
\end{align*}
\begin{align*}
    \forall a, b \in \mathbb{Z},\ b > 0,\ \exists q, r \in \mathbb{Z}\ :\ (a = qb + r) \wedge (0 \leq r < b)
\end{align*}

\section*{Logic}

\textbf{Statement:} A sentence or expression that is definitely true or false.

\textbf{Open sentence:} A sentence whose truth depends on one or more variables.

\begin{center}
    \begin{tabular}{| l | l || l |}
    \hline
    P     & Q     & $P \implies Q$ \\ \hline\hline
    false & false & true           \\ \hline
    false & true  & true           \\ \hline
    true  & false & false          \\ \hline
    true  & true  & true           \\ \hline
    \end{tabular}
\end{center}

\begin{center}
    \begin{tabular}{| l | l || l |}
    \hline
    P     & Q     & $P \iif Q$ \\ \hline\hline
    false & false & true       \\ \hline
    false & true  & false      \\ \hline
    true  & false & false      \\ \hline
    true  & true  & true       \\ \hline
    \end{tabular}
\end{center}

\begin{align*}
    \sim (P \wedge Q) &= (\sim P) \vee (\sim Q) \\
    \sim (P \vee Q) &= (\sim P) \wedge (\sim Q)
\end{align*}
\begin{align*}
    P \implies Q\ =\ (\sim Q) \implies (\sim P)
\end{align*}
\begin{align*}
    \sim (\forall x, P(x))\ &=\ \exists x, \sim P(x) \\
    \sim (\exists x, P(x))\ &=\ \forall x, \sim P(x)
\end{align*}
\begin{align*}
    \sim (P \implies Q)\ = P \wedge \sim Q
\end{align*}

\begin{center}
\begin{tabular}{l}
    $P \implies Q$ \\
    $P$ \\ \hline
    $Q$
\end{tabular}
\qquad
\begin{tabular}{l}
    $P \implies Q$ \\
    $\sim Q$ \\ \hline
    $\sim P$
\end{tabular}
\qquad
\begin{tabular}{l}
    $P \vee Q$ \\
    $\sim P$ \\ \hline
    $Q$
\end{tabular}
\end{center}

\section*{Counting}

\begin{center}
    \begin{tabular}{|c||c|c|}
        \hline
        Replacement & Ordered          & Unordered          \\ \hline\hline
        With        & $n^r$            & $\C{n + r - 1}{r}$ \\ \hline
        Without     & ${}_n P_r$, $n!$ & $\C{n}{r}$         \\ \hline
    \end{tabular}
\end{center}

\begin{align*}
    _n P_r &= \frac{n!}{(n - r)!} \\
    _n C_r &= \frac{_n P_r}{r!}
\end{align*}
\begin{align*}
    X \subseteq B \implies |B - X| = |B| - |X|
\end{align*}
\begin{align*}
    \C{n + 1}{k} = \C{n}{k - 1} + \C{n}{k}
\end{align*}
\begin{align*}
    \C{n}{k} = \C{n}{n - k}
\end{align*}
\begin{align*}
    |A \cup B| = |A| + |B| - |A \cap B|
\end{align*}

\begin{samepage}
Suppose $n$ objects go in $k$ boxes $B_1, B_2, \cdots, B_k$.
\begin{itemize}
    \item $\exists B_m : |B_m| \geq \lceil \frac{n}{k} \rceil$
    \item $\exists B_m : |B_m| \leq \lfloor \frac{n}{k} \rfloor$
    \item $n > k \implies \exists B_m : |B_m| > 1$
    \item $n < k \implies \exists B_m : |B_m| = 0$
\end{itemize}
\end{samepage}

\section*{Proof}

\subsection*{Direct Proof}
\textit{Proposition:} $P \implies Q$.

\textit{Proof:} Suppose $P$.

$\vdots$

Therefore $Q$.

\subsection*{Cases}
\textit{Proposition:} $P \implies Q$.

\textit{Proof:} Suppose $P$. Then $R$, $S$, or $T$. Let's consider these cases separately.

\textbf{Case 1:} \ldots

\textbf{Case 2:} \ldots

\textbf{Case 3:} \ldots

These cases show that $Q$.

\subsection*{Contrapositive proof}
\textit{Proposition:} $P \implies Q$

\textit{Proof:} Suppose $\sim Q$.

$\vdots$

Therefore $\sim P$.

\subsection*{Proof by contradiction}
\textit{Proposition:} $P \implies Q$.

\textit{Proof:} Assume $P$ and $\sim Q$.

$\vdots$

Therefore $C \wedge \sim C$.

\subsection*{Biconditional statements}
\textit{Proposition:} $P \iif Q$.

\textit{Proof:} Prove both $P \implies Q$ and $Q \implies P$.

\subsection*{Equivalent statements}
\begin{align*}
    a & \implies b \\
    \Uparrow & \qquad \: \Downarrow \\
    f & \: \Longleftarrow \ d
\end{align*}

\subsection*{Existence statements}
\textit{Proposition:} $\exists x, P(x)$.

Show that the statement is true by using some value of $x$ as an example. This is a \textit{constructive} proof. If you do not show a witness that proves the statement, the proof is \textit{non-constructive}.

\subsection*{Proofs involving sets}

\subsubsection*{Set inclusion}
\begin{align*}
    P(a) \implies a \in {x:P(x)}
\end{align*}
\begin{align*}
    (a \in S) \wedge P(a) \implies a \in {x \in S : P(x)}
\end{align*}

\subsubsection*{$A \subseteq B$ direct}

\textit{Proof:} Suppose $a \in A$.

$\vdots$

Therefore $a \in B$. Thus $a \in A$ implies $a \in B$, so it follows that $A \subseteq B$.

\subsubsection*{$A \subseteq B$ contrapositive}

\textit{Proof:} Suppose $a \notin B$.

$\vdots$

Therefore $a \notin A$. Thus $a \notin B$ implies $a \notin A$, so it follows that $A \subseteq B$.

\subsubsection*{$A = B$ direct}
\textit{Proof:} Prove both that $A \subseteq B$ and $B \subseteq A$. Therefore, since $A \subseteq B$ and $B \subseteq A$, it follows that $A = B$.

\section*{Disproof}

In general, to disprove $P$, prove $\sim P$.

\subsection*{Counterexample}

To disprove $\forall x \in S, P(x)$, give an example $x \in S$ such that $\sim P(x)$.

To disprove $P(x) \implies Q(x)$, give an example $x$ such that $P(x) \wedge \sim Q(x)$.

\subsection*{Contradiction}

Show that $P \implies C \wedge \sim C$.

\section*{Induction}

Consists of a \textit{basis step} and an \textit{inductive step}. The assumption that $S_k$ is true in the inductive step is the \textit{inductive hypothesis}. To apply induction for all $\mathbb{Z}$, we must show $(S_k \implies S_{k+1}) \wedge (S_k \implies S_{k-1})$. An occasionally convenient alternative to $S_k \implies S_{k+1}$ is $S_{k-1} \implies S_k$.

\subsection*{Induction outline}

\textit{Proposition:} The statements $S_1, S_2, S_3, \ldots$ are all true.

\textit{Proof:} (Induction)
\begin{enumerate}
    \item Prove that the first statement $S_1$ is true.
    \item Given any integer $k \geq 1$, prove that $S_k \implies S_{k + 1}$.
\end{enumerate}
It follows by mathematical induction that every $S_n$ is true.

\subsection*{Strong induction outline}
Useful when $S_k \implies S_{k+1}$ is difficult to establish.

\textit{Proposition:} The statements $S_1, S_2, S_3, S_4, \ldots$ are all true.

\textit{Proof:} (Strong induction)
\begin{enumerate}
    \item Prove the first statement $S_1$, or the first several $S_n$ as needed.
    \item Given any integer $k \geq 1$, prove $(S_1 \wedge S_2 \wedge S_3 \wedge \cdots \wedge S_k) \implies S_{k+1}$.
\end{enumerate}

\section*{Relations}

\begin{align*}
    R \subseteq A \times B
\end{align*}
\begin{align*}
    &\textbf{Reflexive:} & &\forall x \in A,\ xRx \\
    &\textbf{Symmetric:} & &\forall x,y \in A,\ xRy \implies yRx \\
    &\textbf{Transitive:} & &\forall x,y,z \in A, (xRy \wedge yRz) \implies xRz
\end{align*}
\textbf{Equivalence relation:} Reflexive, symmetric, and transitive.
\begin{align*}
    [a] = \curly{x \in A : xRa}
\end{align*}
\textbf{Partition:} A division of a set into non-empty, disjoint subsets, whose union is the set itself.
\begin{align*}
    [a] +      [b] &= [a + b] \\
    [a] \times [b] &= [a \times b] \\
\end{align*}

\section*{Functions}
$f: A \rightarrow B$ is a relation $f \subseteq A \times B$ such that, for all $a \in A$, there exists exactly one tuple $(a, b)$.

$(a,b) \in f \iif f(a) = b$

Range is a subset of the codomain.

\begin{align*}
    &\textbf{Injective:} & &\forall a,b \in A,\ a \neq b \implies f(a) \neq f(b) \\
    &\textbf{Surjective:} & &\forall b \in B,\ \exists a \in A : f(a) = b
\end{align*}
\textbf{Bijective:} Both injective and surjective.

Injective means one-to-one, no output is repeated.

Surjective means onto, the codomain is saturated.

Given finite sets $A$ and $B$ and the function $f: A \rightarrow B$,
\begin{enumerate}
    \item $|A| > |B| \implies f$ is not injective
    \item $|A| > |B| \implies f$ is not surjective
\end{enumerate}

Functions are associative but not commutative.

\begin{align*}
    R^{-1} = {(y,x) : (x,y) \in R}
\end{align*}
\begin{align*}
    i_A (x) = x
\end{align*}
\begin{align*}
    (f \text{ is bijective}) \iif (f^{-1} \text{ is a function})
\end{align*}
Given $f: A \rightarrow B$,
\begin{align*}
    f \circ f^{-1} &= i_A \\
    f^{-1} \circ f &= i_B \\
\end{align*}
Given $f: A \rightarrow B$ and $X \subseteq A$,
\begin{align*}
    f(X) = \curly{f(x) : x \in X}
\end{align*}
Given $f: A \rightarrow B$, $X \subseteq A$, and $Y \subseteq B$,
\begin{align*}
    &\textbf{Image:} & f(X) &= \curly{f(x) : x \in X} \subseteq B \\
    &\textbf{Preimage:} & f^{-1}(Y) &= \curly{x \in A : f(x) \in Y} \subseteq A
\end{align*}
Image means the destination of $X$ under $f$.

Preimage means the source of $Y$ under $f$.

Given $f: A \rightarrow B$, $W, X \subseteq A$, and $Y, Z \subseteq B$, then
\begin{align*}
    f(W \cap X) &\subseteq f(W) \cap f(X) \\
    f(W \cup X) &= f(W) \cup f(X) \\
    X &\subseteq f^{-1}\round{f(X)} \\
    f^{-1}(Y \cup Z) &= f^{-1}(Y) \cup f^{-1}(Z) \\
    f^{-1}(Y \cap Z) &= f^{-1}(Y) \cap f^{-1}(Z) \\
    f\round{f^{-1}(Y)} &\subseteq Y
\end{align*}

\subsection*{Injective direct proof}
Suppose $a, a' \in A$ and $a \neq a'$.

$\vdots$

Therefore $f(a) \neq f(a')$.

\subsection*{Injective contrapositive proof}
Suppose $a, a' \in A$ and $f(a) = f(a')$.

$\vdots$

Therefore, $a = a'$.

\subsection*{Surjective proof}
Suppose $b \in B$.

Show that $\exists a \in A : f(a) = b$.

\end{multicols*}
\end{document}