\documentclass[12pt]{article}

\usepackage{fouriernc}
\usepackage[T1]{fontenc}
\usepackage{amsmath}
\usepackage{amssymb}
\usepackage[utf8]{inputenc}
\usepackage[english]{babel}
\usepackage{multicol}
\usepackage[margin=0.5in]{geometry}

\setlength{\parindent}{0em}
\setlength{\parskip}{1em}

\newcommand{\curly}[1]{\left\{ #1 \right\}}
\newcommand{\soft}[1]{\left( #1 \right)}
\newcommand{\hard}[1]{\left[ #1 \right]}

\title{Formal Languages and Functional Programming Notes}
\author{Tim Harding}

\begin{document}
\begin{multicols*}{2}

\section*{Sets}

\begin{align*}
    A \subseteq B \quad \rightarrow \quad \forall a \in A, a \in B
\end{align*}
\begin{align*}
    A \times B = \curly{(a,b) : (a \in A) \wedge (b \in B)}
\end{align*}
\begin{align*}
    |A \times B| = |A| \times |B|
\end{align*}
\begin{align*}
    \mathcal{P}(A) = \curly{X : X \subseteq A}
\end{align*}
\begin{align*}
    |A| = n \quad \rightarrow \quad |\mathcal{P}(A)| = 2^n
\end{align*}
\begin{align*}
    A \cup B &= \curly{x : (x \in A) \vee (x \in B)} \\
    A \cap B &= \curly{x : (x \in A) \wedge (x \in B)} \\
    A - B &= \curly{x : (x \in A) \wedge (x \notin B)}
\end{align*}
\begin{align*}
    \bar{A} = U - A
\end{align*}
\begin{align*}
    \bigcup_{i = 1}^n A_i &= A_1 \cup A_2 \cup \cdots \cup A_n \\
    \bigcap_{i = 1}^n A_i &= A_1 \cap A_2 \cap \cdots \cap A_n
\end{align*}
\begin{align*}
    \forall a, b \in \mathbb{Z},\ b > 0,\ \exists q, r \in \mathbb{Z}\ :\ (a = qb + r) \wedge (0 \leq r < b)
\end{align*}

\end{multicols*}
\end{document}