\documentclass[12pt]{article}

\usepackage{fouriernc}
\usepackage[T1]{fontenc}
\usepackage{amsmath}
\usepackage{amssymb}
\usepackage[utf8]{inputenc}
\usepackage[english]{babel}
\usepackage{multicol}
\usepackage[margin=0.5in]{geometry}

\setlength{\parindent}{0em}
\setlength{\parskip}{1em}

\newcommand{\curly}[1]{\left\{ #1 \right\}}
\newcommand{\round}[1]{\left( #1 \right)}
\newcommand{\hard}[1]{\left[ #1 \right]}
\newcommand{\C}[2]{
    \begin{pmatrix}
        #1 \\ #2
    \end{pmatrix}
}

\title{Theory of Computing Notes}
\author{Tim Harding}

\begin{document}

\begin{center}
    \begin{tabular}{|l|l|l|}
        \hline
        Regular & Finite automaton & Regular expressions \\ \hline
        Context-Free & Pushdown automaton & Grammar \\ \hline
        Decidable & Turing machine & Code \\ \hline
    \end{tabular}
\end{center}

\section*{Regular}
\subsection*{Deterministic finite automaton}
\begin{center}
    \begin{tabular}{|l|l|}
        \hline
        $Q$ & states \\ \hline
        $\Sigma$ & alphabet \\ \hline
        $\delta : Q \times \Sigma \rightarrow Q$ & transition function \\ \hline
        $q \in Q$ & start state \\ \hline
        $M = (Q, \Sigma, \delta, q, F)$ & Finite automaton \\ \hline
    \end{tabular}
\end{center}

\subsection*{Nondeterministic finite automaton}
\section*{Context-Free}
\section*{Decidable}

\end{document}